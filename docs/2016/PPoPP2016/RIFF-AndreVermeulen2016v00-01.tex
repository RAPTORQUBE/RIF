\documentclass{sigplanconf}

% The following \documentclass options may be useful:

% preprint      Remove this option only once the paper is in final form.
% 10pt          To set in 10-point type instead of 9-point.
% 11pt          To set in 11-point type instead of 9-point.
% authoryear    To obtain author/year citation style instead of numeric.

\usepackage{amsmath}


\begin{document}

\special{papersize=8.5in,11in}
\setlength{\pdfpageheight}{\paperheight}
\setlength{\pdfpagewidth}{\paperwidth}

\conferenceinfo{CONF 'yy}{Month d--d, 20yy, City, ST, Country} 
\copyrightyear{2016} 
\copyrightdata{978-1-nnnn-nnnn-n/yy/mm} 
\doi{nnnnnnn.nnnnnnn}

% Uncomment one of the following two, if you are not going for the 
% traditional copyright transfer agreement.

%\exclusivelicense                % ACM gets exclusive license to publish, 
                                  % you retain copyright

%\permissiontopublish             % ACM gets nonexclusive license to publish
                                  % (paid open-access papers, 
                                  % short abstracts)

\titlebanner{Rapid Information Factory}        % These are ignored unless
\preprintfooter{short description of paper}   % 'preprint' option specified.

\title{Rapid Information Factory}
\subtitle{Applying Lean Six Sigma to Parallel Processsing Framework}

\authorinfo{Andreas Francois Vermeulen}
           {University of St Andrew and University of Dundee}
           {afvermeulen@dundee.ac.uk}
\authorinfo{Dr Vladimir Janjic \and Prof Janet Hughes}
           {University of St Andrew \and University of Dundee}
           {jhughes@computing.dundee.ac.uk \and vj32@st-andrews.ac.uk}

\maketitle

\begin{abstract}
This is the text of the abstract.
\end{abstract}

\category{CR-number}{subcategory}{third-level}

% general terms are not compulsory anymore, 
% you may leave them out
\terms
Parallel applications and frameworks, lean six sigma

\keywords
rapid information factory, rif, framework, lean six sigma, heterogeneous computing, parallel, beowulf, cluster, master-slave, pipe-line.

\section{Introduction}
The Rapid Information Factory is a data processing architecture that enables the enhanced processing of data by using a structured and highly optimised parallel processing framework. The core of the framework is a processing pipeline with feedback to enhance the processing of the the data into information. Our research covers the structure of this processing framework and the use of a three node Beowulf cluster \cite{sonzogni2002parallel} that combines into the formation of the Rapid Information Factory. The improve the performance of the factory we apply same Lean Six Sigma \cite{george2005lean} rules that applies to normal manufacturing factories with proven success.
\section{Research Question}
"Does a Rapid Information Factory improve prosessing of data into information when Lean Six Sigma improvements normally applied to manufactoring factories is used to guide improvements?"
\section{Rapid Information Factory}
The factory is a singular processing solution that processes all data into information using a XML based rules. The solution consists of an interaction between three dimensional frameworks:
\subsection{Customer Framework}
This is the view the single view of the information to the customer. It is structured to show the information in the view the customer spesifies in the functional requirements. As for this spesific research this is not expanded or discussed further.
\subsection{Project Framework} 
The factory evolves over time as it is developed and then redeveloped to handle extra enhancements or new data sources. The project is driven by a agile project methodology consisting a backlog and a cycle of five day sprints covering a period of six months. As for this spesific research this is not expanded or discussed further.
\subsection{Rapid Information Factory Framework}
The farmework is an ontology scripting the processes in Extensible Markup Language (XML) to define the set of rules for encoding the process and the interactions between process. The framework consists of a five high-level layers:
\subsubsection{Business Layer}
The Business Layer contains the Business spesific framework configurations that covers either functional requirements or non-functional requirements. The layer consists of two groupings:
\begin{enumerate}
  \item Functional
  \item Non-functional
\end{enumerate}
As for this spesific research this is not expanded or discussed further.
\subsubsection{Utility Layer}
The Untility Layer consists of sets of utilities for the overall factory. The layer consists of two groupings of utilities:
\begin{enumerate}
  \item Maintance utilities
  \item General utilities
\end{enumerate}
As for this spesific research this is not expanded or discussed further.
\subsubsection{Audit, Balance and Control Layer}
The Audit, Balance and Control Layer (ABC) is the layer that supports the factory while it is running.
This layer handles any active processing allowcated to the Beowulf engine.
The layer consists of three groupings.
\begin{enumerate}
  \item Audit
  \item Balance
  \item Control
\end{enumerate}
\subsubsection{Operational Management Layer}
The Operational Management Layer supports setup of the individual job definition and 
interaction between jobs, job parameters, scheduling, monitoring, communication and alerting within the factory. The layer is the central management engine of the factory. The layer consists of five groupings:
\begin{enumerate}
  \item Jobs
  \item Schedule
  \item Monitor
  \item Communication
  \item Alerts
\end{enumerate}
\subsubsection{Functional Layer}
The functional layer store the scripts that describes every functional process of the complete factory.
The layer consists of six groupings of jobs:
\begin{enumerate}
  \item Retrieve
  \item Assess
  \item Process
  \item Transform
  \item Organise
  \item Report
\end{enumerate}
\subsection{Audit, Balance and Control Layer}
The Audit, Balance and Control Layer (ABC) is the layer that supports the factory while it is running.
This layer handles any active processing allowcated to the Beowulf engine. The layer consists of three groupings.
\subsubsection{Audit}

\subsubsection{Balance}

\subsubsection{Control}

\subsection{Operational Management Layer}
The Operational Management Layer supports setup of the individual job definition and 
interaction between jobs, job parameters, scheduling, monitoring, communication and alerting within the factory. The layer is the central management engine of the factory. The layer consists of five groupings:
\subsubsection{Jobs}

\subsubsection{Schedule}

\subsubsection{Monitor}

\subsubsection{Communication}

\subsubsection{Alerts}

\subsection{Functional Layer}
The functional layer store the scripts that describes every functional process of the complete factory. The layer consists of six groupings of jobs:
\subsubsection{Retrieve}

\subsubsection{Assess}

\subsubsection{Process}

\subsubsection{Transform}

\subsubsection{Organise}

\subsubsection{Report}

\appendix
\section{Performance Improvement Results}
The Performance Improvement Results is as follows:
Applying 5S - Sort improvement process to Retrieve Jobs.
Applying 5S - Set-in-Order improvement process to Retrieve Jobs.
Applying 5S - Shine improvement process to Retrieve Jobs.
Applying 5S - Standardise Sort improvement process to Retrieve Jobs.
Applying 5S - Sustain improvement process to Retrieve Jobs.
Applying Lean Waste - Transport to Retrieve Jobs.
Applying Lean Waste - Inventory to Retrieve Jobs.
Applying Lean Waste - Motion to Retrieve Jobs.
Applying Lean Waste - Waiting to Retrieve Jobs.
Applying Lean Waste - Overproduction to Retrieve Jobs.
Applying Lean Waste - Over-processing to Retrieve Jobs.
Applying Lean Waste - Defects to Retrieve Jobs.

\acks

Thank you to Prof Mark Whitehorn and Prof Janet Hughes for her guidance into the Operational Research, Business Intelligence, Data Science and Data Engineering concepts utilised in this research.

% We recommend abbrvnat bibliography style.

\bibliographystyle{abbrvnat}

% The bibliography should be embedded for final submission.

\bibliography{andre}
\softraggedright

\end{document}

%                       Revision History
%                       -------- -------
%  Date         Person  Ver.    Change
%  ----         ------  ----    ------

%  2015.06.01   AV      0.1--1  First draft
