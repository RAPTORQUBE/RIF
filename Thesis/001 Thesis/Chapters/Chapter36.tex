% Chapter Template

\chapter{Rapid Information Factory Framework (RIFF) - Business Layer - Non-Functional Requirements} % Main chapter title

\label{Chapter36} % Change X to a consecutive number; for referencing this chapter elsewhere, use \ref{ChapterX}

\lhead{Chapter 36. \emph{RIFF - BUS - NFR}} % Change X to a consecutive number; this is for the header on each page - perhaps a shortened title

%----------------------------------------------------------------------------------------
%	SECTION 1
%----------------------------------------------------------------------------------------

\section{Main Section 1}

A non-functional requirement is a requirement that specifies criteria that can be used to test the operation of a factory, rather than specific behaviors of the process within the factory.The set of requirements together as a unit describes the factory verification rules.

%-----------------------------------
%	SUBSECTION 1
%-----------------------------------
\subsection{Subsection 1}

The following are types of non-functional requirements:

\begin{itemize} 
  \item Accessibility
  \paragraph{Accessibility is the "ability to access" the factory. The concept focuses on access for people with disabilities through the use of assistive technology.}
  \item Audit and control
  \paragraph{Tracking requirements for audit and control of the factory.}
  \item Availability
  \paragraph{Availability of factory as a factor of its reliability.}
  \paragraph{A service-level agreement (SLA) is a part of a service contract for the factory.}
  \item Backup
  \paragraph{The backup process is part of the Disaster Recovery Plan for the factory.}
  \item Capacity (current and forecast)
  \paragraph{States the capacity of the CPU, GPU, memory and disk storage at different stages.}
  \item Certification
  \paragraph{Certification of the factory can be achieved for several different ISO standards.}
  \paragraph{This proofs that the factory operates at a pre-agreed level of performance.}
  \item Compliance
  \paragraph{Compliance is achieved against a minimum set of criteria for the factory.}
  \paragraph{The following are compliances that could apply. (Data Protection Act 1998, Freedom of Information Act 2000).}
  \item Configuration management
  \paragraph{Configuration management is a engineering process for establishing and maintaining consistency of a factory's performance, functional and physical attributes.}
  \item Dependency on other parties
  \paragraph{Explains the dependency of the factory on other sources of data.}
  \item Deployment
  \paragraph{Explains how the factory deploys new work items.}
  \item Documentation
  \paragraph{Explains the Documentation required for the factory.}
  \item Disaster Recovery
  \paragraph{Disaster recovery (DR) is a range of policies and procedures to support the recovery or continuation of the factory.}
  \item Efficiency
  \paragraph{Explain resource consumption for preset load within the factory.}
  \item Effectivenes
  \paragraph{Explain resulting performance in relation to effort within the factory.}
  \item Emotional factors
  \paragraph{The emotional factors on factory processing.}
  \item Environmental protection
  \paragraph{Environmental protection of natural environment. This is important for low power consumption in the factory.}
  \item Escrow
  \paragraph{Source code escrow is the deposit of the source code of the factory with a escrow agent. Escrow is requested by a party licensing software to ensure maintenance of the software.}
  \item Exploitability
  \paragraph{Explains how the factory's results is used to achieve extra value.}
  \item Extensibility
  \paragraph{The extensibility of the factory indicates how easy it must be to add new features or data to the existing factory.}
  \item Failure management
  \paragraph{When a node in a factory fails, the strategy of process isolation keeps the factory operational.}
  \item Fault tolerance 
  \paragraph{A fault-tolerant design support the factory to continue intended operation using Operational System Monitoring, Measuring, and Management.}
  \item Legal and licensing issues
  \paragraph{Legal and licensing requirement for the factory.}
  \item Patent-infringement-avoidability
  \paragraph{Requirements for patent.}
  \item Interoperability
  \paragraph{The factory must support interoperability between factory and other systems.}
  \item Maintainability
  \paragraph{The maintainability determines a system of continuous improvement.}
  \item Modifiability
  \paragraph{The factory must support modifiability of ANT and PUPA.}
  \item Network topology
  \paragraph{The network topology is the requirements for the 3D Torus configuration.}
  \item Open source
  \paragraph{Open source is a development model promotes a universal access in the factory.}
  \item Operability
  \paragraph{Operability is the ability to keep factory in a safe and reliable functioning condition.}
  \item Performance / Response time
  \paragraph{Short response time for a spesific PUPA to complete.}
  \paragraph{High throughput in the factory}
  \paragraph{Low utilization of computing resources in the factory.}
  \paragraph{High availability of the factory}
  \item Platform compatibility
  \paragraph{The requirements for Platform compatibility between heterogeneous systems.}
  \item Price/Cost
  \paragraph{Processing cost money the cost in energy to process the data must be minimum.}
  \item Privacy
  \paragraph{The factory's control over information policy.}
  \item Portability
  \paragraph{Usability of the PUPA in different ANT environments.}
  \item Quality 
  \paragraph{Describes the faults discovery, faults delivery, fault removal efficacy and fault reporting of the factory.}
  \item Recovery / Recoverability
  \paragraph{Determines the mean time to recovery (MTTR) of the factory.}
  \item Reliability
  \paragraph{Determines the mean time between failures (MTBF) or availability of the factory.}
  \item Reporting
  \paragraph{The report requirements of the factory.}
  \item Resilience
  \paragraph{Resiliency is the ability to provide and maintain the required level of processing not withstanding any faults and challenges to normal operation of the factory.}
  \item Resource constraints
  \paragraph{The factory will be constraint by current achievable processor speed, memory, disk space and network bandwidth.}
  \item Response time
  \paragraph{Requirement for timely response from the factory.}
  \paragraph{The Tact Time of the factory indicates the requirement.}
  \item Reusability
  \paragraph{Reusability requiremensts in the factory idicates how much of the ANTs and PUPAs can be shared between factories.}
  \item Robustness
  \paragraph{Ability of a factory to handle errors during execution.}
  \item Safety
  \paragraph{The safety factor (SF), is a indication the extra capacity of the factory.}
  \item Scalability
  \paragraph{The horizontal and vertical scalability is the ratio the factory can expand its resources for processing.}
  \paragraph{The algorithm design is a major scalability factor in the factory.}
  \item Security
  \paragraph{Security is the degree of resistance to damage in the factory.}
  \item Software tools
  \paragraph{Explains the software tools that needs support form the factory.}
  \item Stability
  \paragraph{The stability of the factory is state as through the continues support of the processing.}
  \item Standards
  \paragraph{Standards are highly required in the factory to enhance the deep learning output requirements.}
  \item Supportability
  \paragraph{The ability of technical support the factory}
  \item Testability
  \paragraph{Testability is the degree to which a PUPA or ANT can be tested using SMART criteria.}
  \item Usability by user community
  \paragraph{Usability is the level that the outcome of the system support the users own requirements.}
  \item User Friendliness
  \paragraph{The user friendliness is the degree of ease to work with the factory.}
\end{itemize}

%-----------------------------------
%	SUBSECTION 2
%-----------------------------------

\subsection{Subsection 2}

???

%----------------------------------------------------------------------------------------
%	SECTION 2
%----------------------------------------------------------------------------------------

\section{Main Section 2}

???
